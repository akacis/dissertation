\documentclass[doctor]{snuee}

% 논문제목을 넣습니다.
% 필히 한글제목과 영문제목 모두 넣어야 합니다.
\title[korean]{의료용 신호를 위한 뉴럴네트워크 구조와 훈련방식 연구}
\title[english]{Neural Network Model and Training for Medical Sensor Signal}

% 저자 정보를 넣습니다.
% 국문성명, 영문성명 모두 넣어야 하며, 특히 국문성명의 경우는 글자사이에 space가 있는 것과 없는 것
% 두 가지 모두를 집어넣어줘야 합니다.
\author[korean]{최 익 수}
\author[english]{Iksoo Choi}
\author[nospace]{최익수}

% 지도교수님의 성함을 국문으로 넣습니다.
\adviser{심 병 효}

% 논문 제출일, 논문 심사일을 한글로 넣습니다.
\submissiondeadline{2023년 1월} \examinationdate{2022년 12월}

% 졸업일을 영문식, 한글식 두 가지 방법 모두 넣습니다.
\gradyear[english]{FEBRUARY 2023} \gradyear[korean]{2023년 2월}

%

% 문서의 시작
%
% 위의 정보들을 빠짐없이 채워넣고 document를 시작하면
% 외표지, 내표지(외표지와 동일), 인준지가 자동으로 생성됩니다.

\begin{document}
\renewcommand{\baselinestretch}{1.5}    % 본문의 줄간격 조정, 고치거나 삭제하지 마십시오.
\selectfont                             %

% abstract(영문)의 작성
% begin과 end 사이에 abstract의 내용을 채워넣습니다.
\begin{abstract}
	\par %abstract 첫 문장 들여쓰기, 고치거나 삭제하지 마십시오.
	sleepstage, sleepstage sequence modeling, polysomnography
	
	% abstract page 하단에 keyword와 학번을 넣습니다.
	\vfill
	\begin{minipage}[t][20mm][b]{\textwidth}
		{\bfseries keywords}: SNU template, TeX\\
		{\bfseries student number}: 2014-21751\\
	\end{minipage}
	
\end{abstract}

\changepage{5mm}{}{}{}{}{}{}{}{-5mm}    %%페이지 여백 재설정. 절대 고치거나 삭제하지 마십시오.
\makelists   %목차를 자동생성합니다.


% 본문의 시작
% chapter, section의 추가,변경등 모두를 자유롭게 할 수 있습니다.
% 그림, 표의 추가 형식은 MS Word의 형식과 동일합니다. (MS Word Description file 참조)
%

\chapter{INTRODUCTION}
\section{PSG background}
\section{Characteristic of each PSG signal}
\section{Heart Rate Variable from ECG}
\section{background Deep Neural Network Architecture}
\section{contribution}


\chapter{TRANSFORMER BASED SLEEP STAGE CLASSIFICATION MODEL}
\section{icassp paper}
\section{needs for better performance}


\chapter{REGULARIZE METHOD: S-SGD}
\section{S-SGD paper}
\section{apply to model and results}

\chapter{SLEEP STAGE MODEL}
\section{character level ~}
\section{beamsearch withsleepstage model as ASR dose}
\section{model comparison: n-gram, lstm, TF }
\section{cascade decoding}

\input{chapter_5/chap5_conclusion.tex}
%
% 참고문헌을 넣습니다.
% 샘플의 형식과 같은 차례대로 써줍니다.
%

%

\begin{thebibliography}{00}
	
	% 영문저널의 경우
	\bibitem{ref1} B. Jeon and J. Jeong, ``Blocking artifacts
	reduction in image compression with block boundary discontiunity
	criterion,'' {\em IEEE Transactions on Circuits and Systems for
		Video Tech.}, vol. 8, no.3, pp. 345-357, June 1998.
	
	% 영문학술대회의 경우
	\bibitem{ref2} W. G. Jeon and Y. S. Cho, ``An equalization
	technique for OFDM and MC-CDMA in a multipath fading channels,''
	in {\em Proceedings of IEEE Conference on Acoustics, Speech and
		Signal Processing}, Munich, Germany, May 1997. pp. 2529-2532.
	
	% 국내저널의 경우
	\bibitem{ref3} 김남훈, 정영철, ``평탄한 통과대역 특성을 갖는
	새로운 구조의 광도 파로열 격자 라우터,'' {\em 전자공학회논문지},
	제35권 D편, 제3호, 56-62쪽, 1998년 3월.
	
	% 국내학술대회의 경우
	\bibitem{ref4} 윤남국, 김수종, ``무선 센서 네트워크에서의 에너지
	효율적인 그라디언트 기반 라우팅 기법,'' {\em 한국정보과학회
		2006년 추계학술대회}, 제12권, 제2호, 2006년 10월. pp.
	1372-1374.
	
	% 단행본의 경우
	\bibitem{ref5} C. Mead and L. Conway, {\em Introduction to VLSI
		Systems}, Addison-Wesley, Boston, 1994.
	
	% URL
	\bibitem{ref6} The SolarMESH Network,
	http://owl.mcmater.ca/solarmesh
	
	% Technical Report의 경우
	\bibitem{ref7} K. E. Elliott and C. M. Greene, ``A local adaptive
	protocol,'' Argonne National Laboratory, Argonne, France,
	Technical Report 916-1010-BB, 1997.
	
	% 학위논문의 경우
	\bibitem{ref8} T. Kim, ``Scheduling and Allocation Problems in
	High-level Synthesis,'' Ph. D. Dissertation, ECE Department,
	Univ. of Illinois at U-C, 1993.
	
	% 특허의 경우
	\bibitem{ref9} Sunghyun Choi, ``Wireless MAC protocol based on a
	hybrid combination of slot allocation, token passing, and
	polling for isochronous traffic,'' U.S. Patent No. 6,795,418,
	September 21, 2004.
	
	% 표준
	\bibitem{ref10} IEEE Std. 802.11-1999, Part 11: Wireless LAN
	Medium Access Control (MAC) and Physical Layer (PHY)
	specifications, Reference number ISO/IEC 8802-11:1999(E), IEEE
	Std. 802.11, 1999 edition, 1999.
	
\end{thebibliography}

%---------------------------------------------------------------------------------
% 본문의 종료
% 국문 초록과 감사의 글(선택)을 넣습니다.
% begin{summary}와 end{summary}의 사이에 국문초록을 집어넣습니다.
% 감사의글은 \acknowledgement{}의 대괄호 안에 내용을 넣습니다.
%
\begin{summary}
	\par    %첫 줄 들여쓰기를 위한 단락구분.삭제하지 마십시오.
	Put your abstract in Korean.
	\vfill
	\begin{minipage}[t][20mm][b]{\textwidth}
		{\bfseries 주요어}: 서울대학교 논문양식, TeX\\
		{\bfseries 학번}: 2007-2007\\
	\end{minipage}
\end{summary}
\changepage {15mm}{}{}{}{}{-30mm}{}{}{15mm} %초록과 감사의 글을 위한 여백 재설정, 고치거나 삭제하지 마십시오.
\acknowledgement{
	\par
	Put your acknowledgement here.(optional)} %감사의 글을 작성하지 않을경우 삭제가능
\end{document}
